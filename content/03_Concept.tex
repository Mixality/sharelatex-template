\chapter{Konzeption}

Im Kapitel {\em Konzeption} beginnt jetzt die Darstellung der eigenen Arbeit. Dieses Kapitel stellt das Bindeglied zwischen Zielsetzung, Stand der Forschung und der eigentlichen Umsetzung dar. Zusammen mit Implementierung und Evaluation macht die Konzeption den Hauptteil der Arbeit aus.

Typischer Umfang der Konzeption: 5-10 Seiten BA, 15-20 Seiten MA.

\section{Konzeption oder Implementierung}

Vielen fällt es schwer, zwischen Konzeption und Implementierung zu unterscheiden. Das ist auch gar nicht so einfach, aber ein paar Leitlinien können die Entscheidung leichter machen:

\begin{itemize}
    \item Die Konzeption sollte von konkreten technischen Details abstrahieren und sich auf das Kernproblem fokussieren.
    \item Handelt es sich z.B. im Kern um einen Algorithmus, würde man diesen in der Konzeption auf einer abstrakten Ebene beschreiben (z.B. Pseudocode) und in der Implementierung dann auf konkrete Aspekte und Lösungen mittels bestimmter Programmiersprachen, -methoden, oder Bibliotheken eingehen.
    \item In anderen Fällen geht es vielleicht um eine Architektur, dann würde diese mit Mitteln wie z.B. UML in der Konzeption entwickelt und dann in der Implementierung im Detail dargestellt werden.
    \item Steht man vor der Entscheidung, ob etwas eher in die Konzeption oder in die Implementierung gehört, dann kann man dies oft über die folgende Frage entscheiden: {\em Muss jemand, der ein ähnliches Problem unter anderen Bedingungen bearbeitet, das wissen?} Wenn ja, dann gehört es eher in die Konzeption, falls es sich um etwas handelt, was man nur für die konkrete Anwendung wissen muss, dann gehört es eher in die Implementierung. In die Konzeption gehören eher die Aspekte, die etwas abstrakter sind und die wahrscheinlich länger von Bedeutung sind. In die Implementierung gehört alles das, was zum konkreten Verständnis der eigenen Umsetzung notwendig ist und was sich nicht aus der abstrakten Sicht erschließt.
\end{itemize}
