\chapter{Einleitung}

Diese \LaTeX-Vorlage soll als Leitlinie und Anhaltspunkt für das Schreiben von Abschlussarbeiten dienen. Zum einen zeigt die Vorlage eine typische Struktur einer Abschlussarbeit auf, an der sich viele Arbeiten orientieren. Zum anderen gibt diese Vorlage gleichzeitig Tipps und Hinweise sowohl was die Strukturierung als auch was mögliche Technologien, insbesondere \LaTeX-Pakete, angeht.

Typischer Umfang der Einleitung: 1-3 Seiten.

\section{Motivation}

Der wesentliche Abschnitt der Einleitung ist die Motivation, in der die Arbeit in einen Kontext gesetzt wird. Wichtig ist dabei vor allem, den Leser abzuholen und relativ zügig in das Thema einzuführen. Dazu eigenen sich oft auch Bilder oder Skizzen sehr gut. In vielen Fällen kann dies durch ein konkretes Beispiel gelingen.

Die Motivation darf dabei ruhig auf eine größere Vision hin abzielen, auch wenn im Rahmen der Abschlussarbeit dann vielleicht nur ein bestimmter Aspekt oder eine prototypenhafte Umsetzung realisiert werden kann. Es ist aber wichtig, dem Leser den Gesamtkontext zu vermitteln, damit dann die einzelne Leistung besser eingeordnet werden kann.

\section{Zielsetzung}

In diesem Abschnitt sollte die Zielsetzung der Arbeit genau entwickelt werden. Hier bietet es sich an, die Zielsetzungen z.B. als Aufzählung an dieser Stelle einzuführen, damit später auf die einzelnen Zielsetzungen verwiesen werden kann.

\begin{enumerate}[labelsep=1ex]
    \renewcommand{\labelenumi}{\textbf{Z\theenumi.}}
    \item Diese Vorlage soll den grundlegenden Aufbau einer Abschlussarbeit aufzeigen.
    \item Diese Vorlage soll relevante \LaTeX-Beispiele für die eigene Arbeit liefern.
\end{enumerate}

\subsection{Hop oder Top?}
Eine grundsätzliche Frage, die sich bei allen Projekten stellt, ist: wie kann man den Erfolg eines Projektes bestimmen? Für das eigene Arbeiten ist es hilfreich, sich darüber auch beim eigenen Projekt frühzeitig Gedanken zu machen und die Zielsetzungen auch so zu formulieren, dass man es später leicht, den Erfolg zu prüfen.

Eine Formulierung wie {\em 'Es soll ein intuitives Benutzerinterface für gestenbasierte Interaktion entwickelt werden.'} klingt zwar cool, lässt aber die Frage nach der Bestimmung des Erfolgs offen. Eigenschaften wie {\em intuitiv} oder {\em natürlich} zu messen fällt relativ schwer. Eine bessere Variante könnte sein: 

{\em 'Es soll ein intuitives Benutzerinterface für gestenbasierte Interaktion entwickelt werden, dass sich dadurch auszeichnet, dass die Handformen während der gesamten Interaktion richtig bestimmt werden können und eine geringere Anzahl an Fehlinterpretationen erreicht wird als mit dem vom Hersteller mitgelieferten Algorithmus.'}

In dieser Variante wird zum einen die Methode eines Vergleichs mit einem existierenden Ansatz genutzt, zum anderen eine Beschreibung der Abdeckung einer vorgegebenen Situation. Andere Methoden zur Bewertungen können Usability-Tests sein, oder bestimmte Performanz-Maße (höher, schneller, weiter).

Ein wichtiger Vorteil von einer solchen Quantifizierung einer Aufgabenstellung ist es, dass man auch bereits bei der eigenen Entwicklung kontinuierlich messen kann, wie gut das entwickelte System aktuell dasteht.

Von der möglichen Messbarmachung sollte man sich jedoch auch nicht zu sehr einschränken lassen. Manchmal sind es aber auch einfach nur kleine Formulierungsänderungen, die plötzlich eine Quantifizierung ermöglichen.

\section{Aufbau der Arbeit}

In diesem Abschnitt wird schließlich kurz erklärt, wie der weitere Aufbau der Arbeit ist. Welche Kapitel kommen jetzt noch und mit welchem Thema beschäftigen sich diese? Damit soll dem Leser ein kurzer Überblick gegeben werden. Insbesondere bei einer Bachelorarbeit sollte dieser Abschnitt jedoch sehr knapp gefasst werden.
