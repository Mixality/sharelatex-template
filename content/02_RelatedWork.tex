\chapter{Stand der Forschung / Stand der Technik}

In diesem Kapitel soll der Bezug der Arbeit zum aktuellen Stand der Forschung, bzw. zum Stand der Technik, je nach Ausrichtung und Schwerpunkt der Arbeit, verdeutlicht werden. Dazu werden entsprechende Vorarbeiten oder alternative Ansätze vorgestellt und diskutiert. Ziel ist es, den Ansatzpunkt der Arbeit genauer zu bestimmen und etwaige Entscheidungen im späteren Verlauf des Textes zu fundieren.

Dieses Kapitel kann je nach Thema der Arbeit {\emph Stand der Forschung} oder
{\emph Stand der Technik} heißen. In jedem Kapitel ist es wichtig, wie hier
geschehen zu Beginn kurz zu erläutern, um was es in diesem Kapitel geht.

Typischer Umfang des Stands der Forschung: 2-4 Seiten BA, 10-15 Seiten MA.

\section{Richtig zitieren}

In den letzten Jahren gab es auch in der Öffentlichkeit einiges an Diskussion zum Thema {\em Richtiges Zitieren}. Hier die wesentlichen Punkte:

\begin{itemize}
     \item Wenn Aussagen gemacht werden, dann sollten diese wenn möglich durch ein Zitat (eine Literaturstelle) belegt werden.
     \item Alle Literaturstellen werden im Anhang im Literaturverzeichnis gesammelt und über einen entsprechenden Befehl im Text einbezogen (Bibtex ist euer Freund).
     \item Der Teil des Textes, auf den sich die Literaturreferenz bezieht, sollte deutlich werden. 
 
    \begin{itemize}
        \item Ist es eine bestimmte Aussage, dann kommt die Referenz ans Ende des Satzes: \glqq Blickbewegungen in realen Umgebungen kann man mittels Bildverarbeitung auf der Szenenkamera quantifizieren~\citep{3}.\grqq\ Alternativ hätte man es auch anders schreiben können: \glqq \citet{3} zeigen, wie man Blickbewegungen in realen Umgebungen mittels Bildverarbeitung auf der Szenenkamera quantifizieren kann.\grqq 
        \item Oder, bei mehreren Teilsätzen: \glqq Während im ersten Fall einfach Boxen gezeichnet werden~\citep{3}, können andere Heatmaps auf 3D Objekten zeichnen~\citep{4}.\grqq bzw. \glqq Während \citet{3}  einfach Boxen zeichnen, zeichnen \citet{4} Heatmaps auf 3D Objekten.\grqq
    \end{itemize}
 
    \item In der Informatik nutzen wir nur in Ausnahmefällen wörtliche Zitate. Wichtiger sind bei uns Zusammenfassungen und Bewertungen der Aussagen anderer.
\end{itemize}

\section{Breit oder Tief?} 

Gerade in diesem Kapitel sehen sich viele Autoren in einem Zwiespalt. Auf der
einen Seite gibt es sehr viele spannende Inhalte, die man sich vielleicht selbst
erst kürzlich erarbeitet hat. Da macht es natürlich Spaß und dient auch dem
eigenen Verständnis, diese hier wieder zu geben. Allerdings schreibt man dann
primär über Dinge, die andere getan haben und nicht über das, was man selbst
geleistet hat. Ein klarer Fokus bei allen Inhalten in der Abschlussarbeit sollte
jedoch auf der {\normalfont \bfseries eigenen Leistung} liegen. 

Die eigene Leistung kann zum Beispiel darin liegen, aus der vorhandenen
Literatur punktgenau heraus zu arbeiten, was die genauen Unterschiede in den Ansätzen
verschiedener relevanter Algorithmen sind, oder wie sich bestehende
Veröffentlichungen kategorisieren lassen, etc. Die eigene Leistung ist also die
Arbeit mit den Texten, das genauere Hinsehen, und weniger das einfache Berichten über die Texte. Dazu werden die Texte exakt referenziert, so dass man eben nicht mehr gezwungen ist, alle auch nur entfernt relevanten Inhalte hier wieder zu geben.

Der Platz in diesem Kapitel sollte also primär für einen guten breiten Überblick dienen, der dann ins Detail geht, wenn die Stellen behandelt werden, an denen die vorliegende Arbeit ganz konkret ansetzt bzw. für die die vorliegende Arbeit Alternativen erarbeitet.

\section{Zusammenfassung} 

Jedes Kapitel sollte mit einer eigenen Zusammenfassung abschließen (vielleicht
ausgenommen dem einleitenden Kapitel). Der einleitende Text des Kapitels und die
Zusammenfassung bilden zugleich eine Klammer um das Kapitel und zeigen einen
roten Faden im Übergang zwischen den Kapiteln auf. 

Das wesentliche bei der Zusammenfassung insbesondere im Kapitel {\emph Stand der
Forschung} ist es, das im Kapitel beschriebene in eigenen Worten kurz und prägnant
darzustellen und in Bezug zur eigenen Arbeit zu setzen.

Es könnte in der Zusammenfassung zum Beispiel stehen: \glqq Wie X und Y gezeigt
haben, ist noch offen, wie... In dieser Arbeit soll diese Frage so und so
angegangen werden.\grqq Oder \glqq Wie gezeigt werden konnte, gibt es derzeit für X
noch keine (zufriedenstellende) Lösung...\grqq.
