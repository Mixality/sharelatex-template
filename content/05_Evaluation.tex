\chapter{Evaluation}

Wenn möglich, dann sollte jede Arbeit auch in irgendeiner Form evaluiert werden. Die Art der Evaluation kann sehr unterschiedlich sein, je nach Inhalt. Es kann eine Performanz-Messung sein, eine Nutzer-Studie, ein Vergleich oder auch die qualitative Bewertung dem Leser überlassen (z.B. bei Kunstalgorithmen).

Besonders praktisch ist es natürlich, wenn sich die Evaluation direkt aus den Zielsetzungen in der Einleitung ableiten lässt.

Wichtig ist, dass in der Evaluation erst einmal relativ neutral über die Art der Evaluation und die erzielten Ergebnisse berichtet werden sollte. Eine Bewertung findet erst später in der Diskussion statt.

Typischer Umfang der Evaluation: 2-3 Seiten BA, 5-10 Seiten MA.
